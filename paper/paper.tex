\documentclass[12pt]{article}

\usepackage{amsmath}
\usepackage{amssymb}
\usepackage{graphicx}
\usepackage{hyperref}

\title{Quantum-Inspired Decision-Making Strategies in Classical Craps: A Simulation-Based Comparative Study of Classical, Quantum Rational, and QBist Agents}
\author{Adam Hasler}
\date{\today}

\begin{document}

\maketitle

\begin{abstract}
This paper explores the application of quantum mechanics-inspired decision strategies in the classical gambling game of craps. We develop a simulation framework that models the standard rules of craps and introduces three types of betting agents: a classical rational agent, a quantum rational agent employing subjective probability structures derived from quantum measurement theory, and a QBist agent using personal probability updates. Extensive simulations are conducted to compare agent performance in terms of profitability, risk, and survival probability. The study examines whether quantum-influenced probability reasoning can lead to distinguishable outcomes in a classical stochastic environment.
\end{abstract}

\tableofcontents

\newpage

\section{Introduction}
\begin{itemize}
    \item Motivation for studying the influence of quantum mechanics concepts on decision-making strategies in classical gambling.
    \item Overview of quantum probability frameworks and their potential relevance to decision processes.
    \item Goals of comparing classical rational strategies with quantum-inspired approaches.
    \item Summary of the paper's structure and experimental methodology.
\end{itemize}

\section{Classical Craps Overview}
\begin{itemize}
    \item Basic rules of craps.
    \item Description of key bets: Pass Line, Come bets, Odds bets.
    \item Probability and payout structures.
    \item Statistical properties of craps: house edge, expected values.
\end{itemize}

\section{Quantum Mechanics and QBism: Foundations and Mathematical Structure}
\begin{itemize}
    \item Quantum Probability vs Classical Probability.
    \item Quantum Measurement Theory:
    \begin{itemize}
        \item Projection-valued measures (PVMs).
        \item Positive operator-valued measures (POVMs).
    \end{itemize}
    \item Symmetric Informationally Complete POVMs (SIC-POVMs):
    \begin{itemize}
        \item Definition and properties.
        \item SIC representation of quantum states.
    \end{itemize}
    \item Quantum Bayesianism (QBism):
    \begin{itemize}
        \item Subjective probability in quantum mechanics.
        \item Reinterpretation of the Born rule.
        \item Comparison to Copenhagen and Many-Worlds interpretations.
    \end{itemize}
    \item Mathematical implications for decision-making:
    \begin{itemize}
        \item Coherence, Dutch Book arguments.
    \end{itemize}
\end{itemize}

\section{Simulation Design: Modeling Classical and Quantum-Inspired Agents in Craps}
\begin{itemize}
    \item Overview of experimental goals.
    \item Classical craps engine:
    \begin{itemize}
        \item Implementation notes (minimal restatement of game mechanics).
        \item Validation methods.
    \end{itemize}
    \item Agent models:
    \begin{itemize}
        \item Classical Rational Agent.
        \item Quantum Rational Agent.
        \item QBist Agent.
    \end{itemize}
    \item Parameters of simulation:
    \begin{itemize}
        \item Bankroll limits, bet sizing, number of sessions.
        \item Odds bets configuration.
    \end{itemize}
    \item Output metrics:
    \begin{itemize}
        \item Average ending bankroll.
        \item Probability of bankroll depletion.
        \item Variance and volatility measures.
    \end{itemize}
    \item Summary of experimental hypotheses.
\end{itemize}

\section{Results}
\begin{itemize}
    \item Presentation of results for each agent type.
    \item Comparative analysis:
    \begin{itemize}
        \item Profitability.
        \item Risk and volatility.
        \item Survival rates.
    \end{itemize}
    \item Statistical significance of differences.
\end{itemize}

\section{Discussion}
\begin{itemize}
    \item Interpretation of results.
    \item Implications for quantum-inspired decision-making.
    \item Philosophical reflections on subjective probability in classical systems.
    \item Limitations and sources of error.
    \item Suggestions for future research.
\end{itemize}

\section{Conclusion}
\begin{itemize}
    \item Summary of key findings.
    \item Broader significance for quantum foundations and decision theory.
    \item Closing remarks.
\end{itemize}

\appendix

\section{Appendix}
\begin{itemize}
    \item Sample Python code structure.
    \item Extended data tables and figures.
    \item Additional experimental runs.
\end{itemize}

\bibliographystyle{plain}
\bibliography{bibliography}

\end{document}
